%http://masterrussian.com/proverbs/russian_proverbs_15.htm
%http://tex.stackexchange.com/questions/816/cyrillic-in-latex
%M-x set-input-method RET cyrillic-jcuken RET
%keys-pairs (mapcar* 'cons
%"йцукенгшщзхъфывапролджэячсмитьбюЙЦУКЕНГШЩЗХЪФЫВАПРОЛДЖ\ЭЯЧСМИТЬБЮ№"
%"qwertyuiop[]asdfghjkl;'zxcvbnm,.QWERTYUIOP{}ASDFGHJKL:\"ZXCVBNM<>#")
%M-x toggle-input-method
%M-x global-set-key to make C-\ toggle


%http://www.talkrussian.com/phrases

\documentclass{article}
%\usepackage[T2A,T1]{fontenc}
\usepackage[T2A]{fontenc}
\usepackage[utf8]{inputenc}
\usepackage[russian,english]{babel}
%"йцукенгшщзхъфывапролджэячсмитьбюёЙЦУКЕНГШЩЗХЪФЫВАПРОЛДЖ\ЭЯЧСМИТЬБЮ№"
%"qwertyuiop[]asdfghjkl;'zxcvbnm,./QWERTYUIOP{}ASDFGHJKL:\"ZXCVBNM<>#")

\begin{document}
\section{June 24}
\label{sec:june-24}


Classmates:
\begin{itemize}
\item David
\item Kane
\item Josh
\item Miniot
\item Remy
\item Jacque
\end{itemize}
\newcommand{\inR}[1]{\foreignlanguage{russian}{#1}}

Semi-vowel \inR{й} - yuh

\inR{как его зовут}  What's his name?
\inR{его зовут ...}  His name is

\inR{как её зовут}  What's her name?
\inR{её зовут ...}  Her name is ...

\inR{здравствуйте}  Hello  -- \inR{привет}  very very very informal. don't use.


\inR{очень }

\subsection{Group A Letters}
\label{sec:group-letters}

\begin{itemize}
\item  Look like English and sound somewhat similar\ 

\begin{description}
\item[М,м] M
\item[К,к] K, skin not kin
\item[Т,т] T stop not top
\item[A,a] A as in father for stressed syllable, syllable before stressed syllable as first letter of word.  A as in about elsewhere
\item[O,o] O as in shore stressed syllable. as a in father syllable preceeding stress or as first letter of word. as a in about elsewhere
\end{description}

\item Look like English but sound different \ 
  \begin{description}
  \item[\inR{В,в}] English v
    \item[\inR{Н,н}] English N
    \item[\inR{Р,р}] Flapped r
    \item[\inR{У,у}] Like u in Luke
    \item[\inR{Е,е}] in stressed syllable like ye in yet  at beginning of word and after a vowel; elsewhere similar to i in bit
  \end{description}

\item do not look like English \ 

  \begin{description}
  \item[\inR{Э,э}] like z
  \item[\inR{Б,б}] like b in cab
  \item[\inR{П,п}] like p in spin not pin
  \item[\inR{Г,г}] like g except in \inR{его}
  \item[\inR{Л,л}] like l in dull
  \item[\inR{Ш,ш}] like sh in shore
  \item[\inR{З,з}] like e in bet
  \item[\inR{И,и}] like i in magazine
  \item[\inR{Ё,ё}] like yo in York
  \item[\inR{Ю,ю}] like yu in yule
  \item[\inR{Я,я}] in stressed syllable like ya in yacht when at beginning of word, or after a vowel elsewhere similar to the a in yacht. In an unstressed syllalbe similar to yi in Yiddish; at end of word similar to yu in yuppie; elsewhere as i of bit
  \end{description}

\end{itemize}


% \begin{otherlanguage*}{russian}
% Текст на русском языкеw
% \end{otherlanguage*}
% A word and another \foreignlanguage{russian}{слово}
\end{document}
