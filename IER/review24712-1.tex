\documentclass{article}
\usepackage{authordate1-4}
\usepackage{datetime}
\usepackage{amsmath}
\title{Review 24712-1 ``Solvability of Perturbation Solutions in DSGE Models'' by Hong Lan and Alexander Meyer-Gohde}
%\author{Gary S. Anderson}
\newcommand{\ny}{n_y}
\newcommand{\nz}{(n_y+n_e)}
\begin{document}
\maketitle


\makeatletter
\def\fullpath{\begingroup\everyeof{\noexpand}\@sanitize
  \edef\x{\@@input|"find `pwd` -name \jobname.tex" }%
  \edef\x{\endgroup\noexpand\zap@space\x\noexpand\@empty}\x}
\makeatother



%file located in \footnote{\fullpath}

%mailinglabel
%\cc{}
%\initials{}
%\ps{}
This document is my review of
MS 24712-1 ``Solvability of Perturbation Solutions in DSGE Models'' by Hong Lan and Alexander Meyer-Gohde which has been submitted for publication in 
the {\em International Economic Review.}
The text which follows is meant to\ elaborate upon my {\bf Strong Revise and Resubmit} recommendation.

\begin{description}
\item[Paper Objectives]\ 

 The paper shows that a straightforward application of
linear algebraic techniques can provide proofs of several important properties
of standard perturbation method solution procedures.
They present 
conditions guaranteeing that well known iterative procedures can compute perturbation solutions:
  \begin{itemize}
  \item of arbitrarily high order for stationary models.
  \item to a finite order for non-stationary models.
  \item contain no first order perturbation parameter terms.
\end{itemize}
Their conditions generalize the local existence theorem found in\cite{jin02}.
They provide an example demonstrating that iterative procedures can fail for
non-stationary models.

\item[Key Insights] \ 

The authors 

\begin{itemize}
\item build upon linear algebraic tools presented in\cite{vetter1973} to provide
a concise linear algebraic characterization of the linear systems that
arise in computing higher order perturbation solution coefficients.
\item provide (often constructive) proofs based on properties of 
matrix pencils and Sylvester equations.
\item Provided a good example showing how the iterative procedure for computing
perturbation expansion coefficients can break down.
\end{itemize}

\item[Problems] \ 

Unfortunately, the argument was very difficult to follow. I found the discussion
very disjoint.  It was very difficult to navigate the argument as I had to
often flip back and forth between the main text and (unspecified) sections
of the appendix. 
Since there are so many ``moving parts,'' I think
 it would be useful to provide a paragraph
providing a  concise description of the components of the proof and  how the
various  parts of the proof are (or are not) interconnected.

\item[Recommendation] \ 

I argue for a {\bf Strong Revise and Resubmit} recommendation.  I think the
subject matter is of growing relevance for economic modeling.  It is reassuring
to know that checking stability properties at first order is
sufficient to be sanguine about computing higher order terms for saddle-point
stationary models.  
In addition, providing conditions for computing perturbation solutions
for non-stationary is a useful contribution. My only real complaint is in the
difficulty I had following the text.


\item[Nit Picking Comments] \ 

\newcommand{\sap} {See Annotated PDF}

  \begin{description}
  \item[Abstract] \ 

    \begin{itemize}
    \item \sap
    \end{itemize}

  \item[Introduction] \ 


    \begin{itemize}
    \item \sap
    \item Seems a bit long
    \item perturbation solutions are not always policy functions
    \item ``Whether policy function exists .. not the issue'' 
Meaning the paper assumes they exist and are differentiable?
\item Unless there a distinction to be made between solvability and non-singularity, it may be better to stick to one or the other word.
\item It would be worthwhile to explicitly define the Sylvester Equation form or
explicitly direct the reader to \cite{chu87}
    \end{itemize}

  \item[DSGE Problem Statement and Policy Function]\ 


    \begin{itemize}
    \item \sap
    \item Taylor series shown in main text has a 
derivation given in the appendix as corollary A.2 but never referred to in text.
\item text refers to $R^+$
\item Perhaps all the matrix derivative references should only appear in the
appendix since that's where they are actually used.
\item notation  $y_{z^j\sigma^i}$ is elegant, but never actually defined or 
explicitly related to \cite{vetter1973}. There are other formulations which 
look the same but don't map to the \cite{vetter1973} layout. Perhaps mentioning
matrix dimensions would be helpful.
  \begin{gather*}
    y_{z^j\sigma^i} \text{ is } [\ny \times \ny^j +1][\ny^j +1 \times 1]
  \end{gather*}
\item perhaps useful to talk about conformable matrices as per the Kronecker application of derivatives in equations like equation 9
    \end{itemize}

  \item[Higher Order Perturbation: Existence and Uniqueness]\ 

    \begin{itemize}
    \item \sap
\item Lemma 3.1 refers to coefficients as ``undetermined'', perhaps 
unknown would
be better since by the theorem they are ``determined''
\item Not sure about the need to use real eigenvalues in the theorems since all the algorithms I know of compute using complex eigenvalues and (since as you point out,) they always keep conjugate pairs together can construct real solutions.
\item error page 8 ``constructed using these stable root''
    \end{itemize}
  \item[Solvents, Sylvesters, and Proof of Theorem 3.7]\ 


    \begin{itemize}
    \item \sap

    \item Should define pencils and regularity since they figure so prominently in the proofs
    \item How about factoring over deflating?
    \item Gantmacher references with sections or pages were more helpful than to volume
  \item footnote 23 has appendix reference to pencil, but should come earlier.
It's not clear why this footnote is in the text at the point of reference on page 10.
\item Why mention Viete's formula
\item Could more clearly indicate that the first proofs in section 4.1 are for first order -- pre Sylvester Equations
\item ``adapt'' instead of  ``adopt'' Chu notation?
    \end{itemize}
  \item[Non-stationary Policy Functions]\ 

    \begin{itemize}
    \item \sap
    \item why latent roots real  especially since they are ``inside the unit circle''
    \end{itemize}

  \item[Applications]\ 

    \begin{itemize}
    \item \sap
    \end{itemize}
  \item[Conclusion]\ 

    \begin{itemize}
    \item No comments
    \end{itemize}


  \item[Appendix] \ 

    \begin{itemize}
    \item A.1

      \begin{itemize}
      \item Apparently uses \cite{vetter1973} equation 31 but overloads us of M.
Why include the remainder?
      \end{itemize}

    \item A.2
      \begin{itemize}
      \item \sap
      \item Elevate to main text with regularity definition
      \end{itemize}
    \end{itemize}
  \end{description}
\end{description}


\bibliographystyle{authordate1}
\bibliography{files}
\end{document}
