\documentclass{article}
\usepackage{authordate1-4}
\usepackage{frb}
\usepackage{bookman}
\usepackage{amsmath}
%\spreadout
\myphone{(202) 452-2687}
\myemail{ganderson\char64frb.gov}
\myuucp{}
\myname{Gary S. Anderson}
\myrank{}
\sendto{}


\makeatletter
\def\fullpath{\begingroup\everyeof{\noexpand}\@sanitize
  \edef\x{\@@input|"find `pwd` -name \jobname.tex" }%
  \edef\x{\endgroup\noexpand\zap@space\x\noexpand\@empty}\x}
\makeatother

\begin{document}
%\spreadout
\begin{letter}{To Whom It May Concern:}


%file located in \footnote{\fullpath}

%mailinglabel
%\cc{}
%\initials{}
%\ps{}
This document is the cover letter for my review of
MS 24712-1 ``Solvability of Perturbation Solutions in DSGE Models'' by Hong Lan and Alexander Meyer-Gohde that has been submitted for publication in 
the {\em International Economic Review.}
The text which follows is meant to\ elaborate upon my {\bf Strong Revise and Resubmit} recommendation.

\begin{description}
\item[Paper Objectives] \ 

The paper shows that a straightforward application of
linear algebraic techniques can provide proofs of several important properties
of standard perturbation method solution procedures.
They present 
conditions guaranteeing that well known iterative procedures can compute perturbation solutions:
  \begin{itemize}
  \item of arbitrarily high order for stationary models.
  \item to a finite order for non-stationary models.
  \item contain no first order perturbation parameter terms.
\end{itemize}
Their conditions generalize the local existence theorem found in\cite{jin02}.
They provide an example demonstrating that iterative procedures can fail for
non-stationary models.

\item[Key Insights] \ 

The authors 

\begin{itemize}
\item build upon linear algebraic tools presented in\cite{vetter1973} to provide
a concise linear algebraic characterization of the linear systems that
arise in computing higher order perturbation solution coefficients.
\item provide (often constructive) proofs based on properties of 
matrix pencils and Sylvester equations.
\item Provided a good example showing how the iterative procedure for computing
perturbation expansion coefficients can break down.
\end{itemize}

\item[Problems] \ 

Unfortunately, the argument was very difficult to follow. I found the discussion
very disjoint.  It was very difficult to navigate the argument as I had to
often flip back and forth between the main text and (unspecified) sections
of the appendix. 
Since there are so many ``moving parts,'' I think
 it would be useful to provide a paragraph
providing a  concise description of the components of the proof and  how the
various  parts of the proof are (or are not) interconnected.

\item[Recommendation] \ 

I argue for a {\bf Strong Revise and Resubmit} recommendation.  I think the
subject matter is of growing relevance for economic modeling.  It is reassuring
to know that checking stability properties at first order is
sufficient to be sanguine about computing higher order terms for saddle-point
stationary models.  
In addition, providing conditions for computing perturbation solutions
for non-stationary is a useful contribution. My only real complaint is in the
difficulty I had following the text.

\end{description}
\bibliographystyle{authordate1}
\bibliography{files}

\end{letter}
%\mailinglabel
\end{document}
