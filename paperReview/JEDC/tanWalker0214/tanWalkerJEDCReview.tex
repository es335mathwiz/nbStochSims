\documentclass[12pt]{article}

\usepackage{hyperref}
\hypersetup{
  pdfinfo={
  pdfproducer={},
  Title={},
  Subject={},
  Author={},
  }
}

\usepackage{amsmath}
\usepackage{amssymb}
\usepackage{authordate1-4}
\begin{document}

\centerline{\large\bf Reviewer ``Blind'' Comments to Authors}

\begin{description}
\item[Four Cases] The text identifies four cases for the model of section
4.2.  Although there are, in fact, four cases, the authors appear to 
have either omitted additional constraints on the parameters or to have 
mischaracterized the ranges for $\alpha, \beta, \gamma$ for the cases.  
For example, a single root inside the unit circle requires
\begin{gather*}
(-1<\alpha<1 \wedge \alpha \ne 0) \wedge \left (
\{\gamma > 1\} \cup \left ( \gamma < \frac{(\beta +1)}{(1-\beta )}\right)\right)
\end{gather*}
Neither the authors nor the 
paper they refer to \cite{Leeper91} 
constrains the signs of the $\alpha$ or $\gamma$.

\item[Maple/muPad] Will the authors make the code available?
Matlab has recently replaced Maple with muPad.  What versions of
Matlab will the author's code support?  

\item[Symbolic Computation] The author's development of a symbolic algebra
solution for linear rational expectations models 
is novel and potentially useful.  However, it should be noted that  \href{http://www.federalreserve.gov/pubs/oss/oss4/papers/reliableMath/reliableMath.html}{ a symbolic algebra version of the Anderson-Moore Algorithm (AMA)} has been available for about 10 years.\footnote{Hover over the sentence to follow the link.}

\item[Decision Rule Computation] \cite{Anderson10} provides 
an auto-regressive representation of the solution (often referred to as 
a decision rule representation) as well as a formula for computing impulse 
response functions. It would be useful to  refer 
the reader to techniques for discovering the decision rule
representation for the solutions provided.



\item[Why not Volterra series?] The utility of linear forecasts
for  $y_t$ given the exogenous
process appears to be the key model feature that allows the 
algebra in their proof to work.  The AMA approach relies upon this 
same feature but can accommodate more general exogenous processes.
I suspect that the authors could also modify their
algorithm to accommodate more general exogenous processes.


\item[Useful in Practice?] Although useful in theory, the
proposed algorithm may not be useful in practice.
I implemented the author's approach in Mathematica.\footnote{I used the Elsevier website to ask for code from the authors, but never got a response.}
My implementation of the Tan Walker Approach (TWA)
only computes the C(L)  series  (impulse response functions) 
described in the paper.  I did not experiment with obtaining the decision
rule representation.
As noted above, the Mathematica implementation of AMA 
computes both the ``decision rule'' and the impulse
response function.  The computation time below refers to obtaining both
forms of the solution.\footnote{The formula in \cite{Anderson10} requires
iteration unless the exogenous process can be cast as an AR(1).  I experimented
with calculations just a dozen or so periods ahead.}

When my implementation of their TWA terminates, the TWA and AMA 
solutions agree.
I did not have time to do much experimentation, but for the models I tried,
the symbolic algebra version of AMA was much much faster than the 
symbolic implementation of TWA.
I suspect that the bottleneck for the Tan Walker implementation is 
symbolically computing the Smith Normal Form.  For the ``Asset Pricing Approximation'' model listed below,  the AMA algorithm required 12 seconds of CPU time.
After running for 24 hours, the Mathematica code 
was unable to compute the Smith Normal Form encountered
in that model.\footnote{ The nonlinear model solution I computed
had symbolic place holders for
the steady state linearization point. This may have added significantly to the
execution time for the Smith Normal Form.}


\item[``Indispensable'' seems a bit strong] Although the 
paper is carefully written and provides a novel alternative perspective
to existing solution techniques, I could not conclude that other existing
approaches could not also easily 
deliver solutions for the examples provided in the text.

\item[Unit Roots] It would be worthwhile for the authors 
to comment on how their approach would deal with models that have unit roots.

\item[Redundant Equations] In following the examples, 
it might help the reader to show  
all  the constraints -- even those that turn out not to help 
``pin down'' the solution.


\end{description}
% $F_1=$\input{F1$42}
% $G_0=$\input{G0$42}
% $G_1=$\input{G1$42}

% $A=$\input{AA$42}
% $\phi_0=$\input{phi0$42}
% $\psi_0=$\input{psi0$42}
% $\psi_1=$\input{psi1$42}


% So that
% $H(z)=z(F(z^{-1}) + G(z))$=\input{pre$42}

% and

% $H(z)= U(z)^{-1}$\input{snf$42}$V(z)^{-1}$

% where 

% $U(z)=$\input{uu$42}

% and

% $V(z)=$\input{vv$42}

% The roots are

% \input{jr$42}

% To apply the algorithms in the more recent paper\cite{Anderson10}

% \begin{gather*}
%   \sum_{i=-\tau}^\theta H_ix_{t+i} = \psi z_t
% \end{gather*}

% $H=$\input{hmat}

% Leads to a transition matrix with eigenvalues:

% \input{evls}



% \begin{description}
% \item[Impulse Response Functions] Can use my formulae to compute
%   \begin{gather*}
%     E_t (y_{t+k}) - E_{t-1} y_{t+k} providing the coefficients in the expansion Tan and Walker provide. \\
%   \end{gather*}
% \item[Volterra] do same for volterra (but then may want GIRF )
% \item[Variable order?] IRF's and shocks order
% \item[Check ROC?]
% \item[Smith normal form] available without maple as in mupad?
% \item[Case 1] Both $\frac{1}{\alpha}>1$ and $\frac{\beta}{\beta \gamma -\gamma +1}>1$

% $\tilde{U}(z)=$\input{UTilde1$42}


% $\tilde{U}(0)=$\input{UTilde1Z0$42}
%  Should point out the first condition is for possibility of existence and how relates to $span(A) \subseteq span(R)$
% Can $\psi_i$ ever enter without multiplicities?
% Why just showing $U_{2\cdot}(z)$


% Should use a different letter for the A in the rank condition

% $\mathcal{A}(z)=$\input{bigA1$42}

% $\mathcal{R}(z)=$\input{bigR1$42}

% $\mathcal{Q}=$\input{bigQ1$42}

% $\mathcal{A}(0)=$\input{bigA1Z0$42}
% $\mathcal{R}(0)=$\input{bigR1Z0$42}

% \item[Case 2] $\frac{1}{\alpha}<1$ and $\frac{\beta}{\beta \gamma -\gamma +1}>1$



% $\tilde{U}(z)=$\input{UTilde2$42}


% $\tilde{U}(\frac{1}{\alpha})=$\input{UTilde1Z0$42}
%  Should point out the first condition is for possibility of existence and how relates to $span(A) \subseteq span(R)$
% Can $\psi_i$ ever enter without multiplicities?
% Why just showing $U_{2\cdot}(z)$


% Should use a different letter for the A in the rank condition

% $\mathcal{Q}=$\input{bigQ2$42}


% $\mathcal{A}_0(z)=$\input{bigA2$42}

% $\mathcal{R}_0(z)=$\input{bigR2$42}

% $\mathcal{A}_1(z)=$\input{bigA2$42}

% $\mathcal{R}_1(z)=$\input{bigR2$42}


% $\mathcal{A}_1(0)=$\input{bigA21Z0$42}
% $\mathcal{R}_1(0)=$\input{bigR21Z0$42}


% $\mathcal{A}_2(\frac{1}{\alpha})=$\input{bigA22Z0$42}
% $\mathcal{R}_2(\frac{1}{\alpha})=$\input{bigR22Z0$42}

% \input{eqnsCase2}


% Should use a different C for solution


% $\mathcal{C}=$\input{bigCCase2}
% \end{description}

% \input{bigCSeries2$42}


% $B=$\input{bmat2$42}

% $\phi=$\input{fmat2$42}

% $F=$\input{fmat2$42}


% \begin{gather*}
% x_{t}=B x_{t-1} + \sum_{k=0}^\infty F^k \phi \psi z_{t+s}  
% \end{gather*}






% \begin{description}\item[Wold shocks/ True Shocks] Conflating the two?
% Wold representation is unique linear representation of time series where shocks are linear forecast errors.
% \item[Impulse Response Functions] How GIRF?
% \item[structural interpretation to errors] problematic
% \item[Wold not true representation] Could be non linear or non-invertible
% Useful examples?
% \item[$\epsilon_t$ ] are the $\epsilon_t$ for y the same as for  x.  Impulse response in terms of fundamental or exogenous.
% \item[In examples the A's are just Identity matrix]   $x_t= \epsilon$ inversion easy.  Should point out the necessity of recovering $\epsilon$ for impulse 
% response functions or omit A.

% \item[non identity example]


% \begin{gather*}
%   A_0= I\\
% A_1= -A_0 C_1\\
% A_k= -A_0 C_k- A_1C_{k-1}-\ldots-A_{k-1}C_1
% \end{gather*}
  
% \item[Practical uses for Wold Decomposition]  Are there any.  Wold shocks moving average not same as impulse response function.  Best linear predictor.  Uncorrelated not independent  may be inappropriate for linearized models.
% \item[more recent Anderson Moore paper ] has formulae
% \item[multiplicities] how common how handle. zeroes easy and always irrelevant?
% \item[implementation] issues?
% \item[computation times?] Likely to be slow and often infeasible
% \item[expectational errors unnecessary]
% \item[Adds no new capability]
% \item[Anderson continuous time]
% \item[provide code] 
% \item[for modesl with infinite lags leads infinite dimensional data structures implicit] 
% \href{http://arxiv.org/pdf/1109.0783.pdf}{From }
% We have insisted on numerical examples, but the co-recursive strategy is, of course, domain independent,
% provided that an adequate set of mathematical operation is defined. The last example combined two
% different sorts of sequences.
% It is possible to write such algorithms in a symbolic setting, and we did it in Maple and in MuPAD.
% But the language of these packages is strict; we used unevaluated macros in order to simulate the lazy
% processing, and the “intermediate expression swell disease” is a serious hindrance. Probably the languages
% based on rewriting, such as Mathematica, are better adapted to this kind of manipulations than
% the very procedural Maple, but we haven’t tested it.
% \end{description}

% Requires non-singularity?  Restrictive wrt models?

% \href{http://math.stackexchange.com/questions/222833/smith-normal-form-of-a-polynomial-matrix}{Smith Normal Form Transformations}

% Symbolic solution is a good feature.

% Compare symbolic speed.

% What about unit roots.  How handle them practical and theoretical.

% Guarantee real valued solutions?

% Leeper section 4.4 $\gamma < 0$

% Implicit in equation $|\alpha| > \beta^{-1}$  constantly refer to $|\alpha \beta <1$

% Leeper doesn't constrain the signs either.
% Active and passive regions depend on $\beta$

% For example 

% \begin{tabular}{|l|c|c|c|c|c|c|}
% \hline
%   \multicolumn{1}{|c|}{$|\alpha|\, \lessgtr\,  1,  \,\, |\gamma|\, \lessgtr\,  1$}&
%   \multicolumn{1}{|c|}{$\alpha$}&
%   \multicolumn{1}{|c|}{$\beta$}&
%   \multicolumn{1}{|c|}{$\gamma$}&
%   \multicolumn{1}{|c|}{$\frac{1}{\alpha}$}&
% \multicolumn{1}{|c|}{$\frac{\beta }{\beta  \gamma -\gamma+1}$}&
% \multicolumn{1}{|c|}{Roots Inside}\\
% \hline
% $|\alpha| <1, |\gamma|>1$&.5&.75&100&2.&-0.03&2\\
% \hline
% $|\alpha| <1, |\gamma|>1$&.5&.75&6.96&2.&-1.013&1\\
% \hline
% $|\alpha| >1, |\gamma|>1$&1.5&.75&100&.667&-0.03&3\\
% \hline
% $|\alpha| >1, |\gamma|>1$&1.5&.75&6.96&2.&-1.01&2\\
% \hline
% \end{tabular}

% One root in requires $|\alpha |>\alpha ^2\land \frac{\beta }{|(\beta -1) \gamma+1|}>1$ which can be true for all non zero $\alpha$ such that $-1<\alpha<1$ or for $\alpha < -1$ and $\gamma < -\frac{(\beta+1)}{(\beta-1)}$ or $\gamma>1$




% \appendix
% \begin{gather*}
% Z(a(n)) = \sum_0^\infty  a(n) z^{-n}
% \end{gather*}

% Other definitions ( bilateral ) possible so should include specific

% \begin{verbatim}
% For each question, please use the following scale to answer (place an x in the space provided): 

% "To what extent does the article meet this criterion?"

% 0	Fails by a large amount
% 1	Fails by a small amount
% 2	Succeeds by a small amount
% 3	Succeeds by a large amount
% 4	Not applicable

% The subject addressed in this article is worthy of investigation.
	
% 0 __ 1 __ 2 __ 3 x__ 4 __


% The information presented was new.

% 0 __ 1 __ 2 __ 3 x__ 4 __

% The conclusions were supported by the data.

% 0 __ 1 __ 2 __ 3 x__ 4 __

% Is there a financial or other conflict of interest between your work and that of the authors?

% YES __  NO x__


% Please give a frank account of the strengths and weaknesses of the article:



% \end{verbatim}
\newpage

\centerline{\bf \large An Asset Pricing Approximation Model}

\begin{verbatim}
var  dc, dd, v_c, v_d, x;
varexo e_c, e_x, e_d;

parameters DELTA THETA PSI MU_C MU_D RHO_X LAMBDA_DX;

DELTA=.99;
PSI=1.5;
THETA=(1-7.5)/(1-1/PSI); 
MU_C=0.0015;
MU_D=0.0015;
RHO_X=.979;
LAMBDA_DX=3;

model;
v_c       = DELTA^THETA * exp((-THETA/PSI)*dc(+1) + 
(THETA-1)*log((1+v_c(+1))*exp(dc(+1))/v_c) ) * (1+v_c(+1))*exp(dc(+1));
v_d       = DELTA^THETA * exp((-THETA/PSI)*dc(+1) + 
(THETA-1)*log((1+v_c(+1))*exp(dc(+1))/v_c) ) * (1+v_d(+1))*exp(dd(+1));
dc        = MU_C  + x(-1) + e_c;
dd        = MU_D + LAMBDA_DX*x(-1) + e_d;
x         = RHO_X * x(-1) + e_x;
end;

\end{verbatim}
\bibliography{files,anderson}
\bibliographystyle{authordate4}


\end{document}
